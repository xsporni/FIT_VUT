
\begin{DoxyRefList}
\item[\label{todo__todo000004}%
\hypertarget{todo__todo000004}{}%
Module \hyperlink{group__task1}{task1} ]1.\+1.) Doprogramujte inicializační funkci \hyperlink{student_8h_ac2adb2ba4e748239b9db4d037584d3cc}{phong\+\_\+on\+Init()}. Zde byste měli vytvořit buffery na G\+P\+U, nahrát data do bufferů, vytvořit vertex arrays object a správně jej nakonfigurovat. Do bufferů nahrajte vrcholy králička (pozice, normály) a indexy na vrcholy ze souboru \hyperlink{model_8h}{model.\+h}. Využijte proměnné ve struktuře \hyperlink{structPhongVariables}{Phong\+Variables} (vbo, ebo, vao). Do proměnné phong.\+vbo zapište id bufferu obsahující vertex atributy. Do proměnné phong.\+ebo zapište id bufferu obsahující indexy na vrcholy. Do proměnné phong.\+vao zapište id vertex arrays objektu. Data vertexů naleznete v proměnné \hyperlink{model_8h}{model.\+h}/model\+Vertices -\/ ty překopírujte do bufferu phong.\+vbo. Data indexů naleznete v proměnné \hyperlink{model_8h}{model.\+h}/model\+Indices -\/ ty překopírujte do bufferu phong.\+ebo. Dejte si pozor, abyste správně nastavili stride a offset Vrchol králička je složen ze dvou vertex atributů\+: pozice a normála. Prozatím zkuste pouze nahrát pozice vertexů Použijte D\+S\+A přístup, viz upozornění v závorkách níže {\bfseries Seznam funkcí, které jistě využijete\+:}
\begin{DoxyItemize}
\item gl\+Create\+Buffers (namísto starého gl\+Gen\+Buffers)
\item gl\+Named\+Buffer\+Data (namísto starého gl\+Buffer\+Data a gl\+Bind\+Buffer)
\item gl\+Create\+Vertex\+Arrays (namísto starého gl\+Gen\+Vertex\+Arrays)
\item gl\+Vertex\+Array\+Element\+Buffer
\item gl\+Vertex\+Array\+Vertex\+Buffer (namísto starého gl\+Get\+Attrib\+Location a gl\+Vertex\+Attrib\+Pointer)
\item gl\+Enable\+Vertex\+Array\+Attrib
\item gl\+Bind\+Vertex\+Array 
\end{DoxyItemize}

1.\+2.) Doprogramujte kreslící funkci \hyperlink{student_8h_a53ffbb1a271d285abdaf7a029192f47e}{phong\+\_\+on\+Draw()}. Zde byste měli aktivovat vao a spustit kreslení. Funcke gl\+Draw\+Elements kreslí indexovaně, vyžaduje 4 parametry\+: mode -\/ typ primitia, počet indexů, typ indexů (velikost indexu), a offset. Kreslíte trojúhelníky, počet vrcholů odpovídá počtu indexů viz proměnná \hyperlink{model_8h}{model.\+h}/model\+Indices.~\newline
 {\bfseries Seznam funkcí, které jistě využijete\+:} (-\/ gl\+Bind\+Vertex\+Array)
\begin{DoxyItemize}
\item gl\+Draw\+Elements  
\end{DoxyItemize}
\item[\label{todo__todo000001}%
\hypertarget{todo__todo000001}{}%
Module \hyperlink{group__task2}{task2} ]2.\+1.) Doimplementujte vertex shader. Vašim úkolem je přidat uniformní proměnné pro view a projekční matici. Dále pronásobte pozici vrcholu těmito maticemi a zapište výsledek do gl\+\_\+\+Position. Nezapomeňte, že píšete v jazyce G\+L\+S\+L, který umožňuje práci s maticovými a vektorovými typy. Upravujte phong\+Vertex\+Shader\+Source proměnnou. 

2.\+2.) Ve funkci \hyperlink{student_8h_ac2adb2ba4e748239b9db4d037584d3cc}{phong\+\_\+on\+Init()} získejte lokace přidaných uniformních proměnných pro projekční a view matice. Zapište lokace do příslušných položek ve struktuře \hyperlink{structPhongVariables}{Phong\+Variables}. Nezapomeňte, že lokace získáte pomocí jména proměnné v jazyce G\+L\+S\+L, které jste udělali v předcházejícím kroku. {\bfseries Seznam funkcí, které jistě využijete\+:}
\begin{DoxyItemize}
\item gl\+Get\+Uniform\+Location 
\end{DoxyItemize}

2.\+3.) Upravte funkci \hyperlink{student_8h_a53ffbb1a271d285abdaf7a029192f47e}{phong\+\_\+on\+Draw()}. Nahrajte data matic na grafickou kartu do uniformních proměnných. Aktuální data matic naleznete v externích proměnných view\+Matrix a projection\+Matrix. {\bfseries Seznam funkcí, které jistě využijete\+:}
\begin{DoxyItemize}
\item gl\+Uniform\+Matrix4fv  
\end{DoxyItemize}
\item[\label{todo__todo000002}%
\hypertarget{todo__todo000002}{}%
Module \hyperlink{group__task3}{task3} ]3.\+1.) Upravte vertex shader. Vašim úkolem je přidat druhý vertex atribut -\/ normálu vrcholu. Dále přidejte dvě výstupní proměnné typu vec3 a zapište do nich pozici a normálu vrcholu ve world-\/space. Tyto proměnné budete potřebovat pro výpočet osvětlení ve fragment shaderu. Upravujte phong\+Vertex\+Shader\+Source proměnnou. 

3.\+2.) Upravte fragment shader (proměnná phong\+Fragment\+Shader\+Source). Vašim úkolem je implementovat phongův osvětlovací model. Přidejte dvě vstupní proměnné (typ vec3) stejného názvu, jako nově přidané výstupní proměnné ve vertex shaderu. V jedné obdržíte pozice fragmentu ve world-\/space. V druhé obdržíte normálu fragmentu ve world-\/space. Dále přidejte dvě uniformní proměnné (typ vec3) pro pozici kamery a pro pozici světla. Difuzní barvu materiálu nastavte podle zadání konkrétní skupiny cvičení, ptejte se pokud nevíte jak má materiál vypadat Spekulání barvu materiálu nastavte na vec3(1.\+f,1.\+f,1.\+f) -\/ bílá.~\newline
 Shininess faktor nastavte na 40.\+f.~\newline
 Předpokládejte, že světlo má bílou barvu.~\newline
 Barva se vypočítá podle vzorce d\+F$\ast$d\+M$\ast$d\+L + s\+F$\ast$s\+M$\ast$s\+L.~\newline
 d\+M,s\+M jsou difuzní/spekulární barvy materiálu -\/ vektory.~\newline
 d\+L,s\+L jsou difuzní/spekulární barvy světla -\/ vektory.~\newline
 d\+F,s\+F jsou difuzní/spekulární faktory -\/ skaláry.~\newline
 d\+F lze vypočíst pomocí vztahu clamp(dot(\+N,\+L),0.\+f,1.\+f) -\/ skalární součin a ořez do rozsahu \mbox{[}0,1\mbox{]}.~\newline
 N je normála fragmentu (nezapomeňte ji normalizovat).~\newline
 L je normalizovaný vektor z pozice fragmentu směrem ke světlu.~\newline
 s\+F lze vypočíst pomocí vztahu pow((clamp(dot(\+R,\+L),0.\+f,1.\+f)),s) -\/ skalární součin, ořez do rozsahu \mbox{[}0,1\mbox{]} a umocnění.~\newline
 s je shininess faktor.~\newline
 R je odražený pohledový vektor V; R = reflect(\+V,\+N).~\newline
 V je normalizovaný pohledový vektor z pozice kamery do pozice fragmentu.~\newline
 ~\newline
 Nezapomeňte, že programujete v jazyce G\+L\+S\+L, který zvládá vektorové operace.~\newline
 {\bfseries Seznam užitečných funkcí\+:}
\begin{DoxyItemize}
\item dot(x,y) -\/ funkce vrací skalární součin dvou vektorů x,y stejné délky
\item clamp(x,a,b) -\/ funkce vrací ořezanou hodnotu x do intervalu \mbox{[}a,b\mbox{]}
\item normalize(x) -\/ funkce vrací normalizovaný vektor x
\item reflect(\+I,\+N) -\/ funkce vrací odražený vektor I podle normály N
\item pow(x,y) -\/ funkce vrací umocnění x na y -\/ x$^\wedge$y 
\end{DoxyItemize}

3.\+3.) Ve funkci \hyperlink{student_8h_ac2adb2ba4e748239b9db4d037584d3cc}{phong\+\_\+on\+Init()} získejte lokace přidaných uniformních proměnných pro pozici světla a pro pozice kamery. Zapište lokace do příslušných položek ve struktuře \hyperlink{structPhongVariables}{Phong\+Variables}. Nezapomeňte, že lokace získáte pomocí jména proměnné v jazyce G\+L\+S\+L, které jste udělali v předcházejícím kroku.~\newline
 {\bfseries Seznam funkcí, které jistě využijete\+:}
\begin{DoxyItemize}
\item gl\+Get\+Uniform\+Location 
\end{DoxyItemize}

3.\+4.) Ve funkci \hyperlink{student_8h_ac2adb2ba4e748239b9db4d037584d3cc}{phong\+\_\+on\+Init()} nastavte druhý vertex atribut pro normálu podobně jako pro pozici. Musíte získat lokaci vstupní proměnné ve vertex shaderu, kterou jste přidali v předcházejícím kroku. Musíte správně nastavit stride a offset -\/ normála nemá nulový offset.~\newline
 

3.\+5.) Ve funkci \hyperlink{student_8h_a53ffbb1a271d285abdaf7a029192f47e}{phong\+\_\+on\+Draw()} nahrajte pozici světla a pozici kamery na G\+P\+U. Pozice světla a pozice kamery je v proměnných phong.\+light\+Position a camera\+Position.~\newline
 {\bfseries Seznam funkcí, které jistě využijete\+:}
\begin{DoxyItemize}
\item gl\+Uniform3f nebo gl\+Uniform3fv 
\end{DoxyItemize}
\end{DoxyRefList}