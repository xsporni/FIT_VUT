\hypertarget{index_zadani}{}\section{Zadání cvičení 6. do předmětu I\+Z\+G 2018 -\/ 2019.}\label{index_zadani}
Vašim úkolem je naimplementovat phongův osvětlovací model pomocí Open\+G\+L. Úkol je složen ze tří částí\+:
\begin{DoxyItemize}
\item Inicializace dat na G\+P\+U a spuštění kreslení \hyperlink{group__task1}{první úkol }. 
\item Implementace transformace vrcholů pomocí vertex shaderu \hyperlink{group__task2}{druhý úkol }. 
\item Implementace phongova osvětlení pomocí fragment shaderu \hyperlink{group__task3}{třetí úkol }.  Seznam všech úkolů naleznete zde \hyperlink{todo}{todo.\+html}.
\end{DoxyItemize}

Ve cvičení se seznámíte se základními funkcemi Open\+G\+L a napíšete si jednoduchý kód ve jazyce G\+L\+S\+L. Prerekvizita cvičení je 11. přednáška o osvětlování a základech Open\+G\+L. Cvičení taktéž kopíruje zadání projektu, ve kterém je úkol totožný, ale nevyužívá se Open\+G\+L.

V projektu je přítomen i příklad vykreslení jednoho čtverce\+: v \hyperlink{quadExample-example}{Quad Example}. Tento příklad můžete využít pro inspiraci a návod jak napsat cvičení, musíte však zaměnit některé funkce za tzv. Direct State Access (D\+S\+A) alternativy (N\+U\+T\+NÉ P\+O\+UŽÍ\+V\+A\+T M\+I\+N. O\+P\+E\+N\+G\+L 4.\+5). Všechny principy Open\+G\+L, které budete ve cvičení používat jsou tam použity (tvorba bufferů, tvorba vertex arrays objektu, indexování, uniformní proměnné, vertex atributy, přeposílání dat mezi vertex shaderem a fragment shaderem).

Všechny úkoly jsou v souboru {\bfseries \hyperlink{student_8c}{student/student.\+c}}. Model králička je v souboru {\bfseries \hyperlink{model_8h}{student/model.\+h}}.\hypertarget{index_teorie}{}\section{Teorie}\label{index_teorie}
Typické grafické A\+P\+I (Open\+G\+L/\+Vulkan/\+Direct\+X) je složeno ze 2 částí\+: C\+P\+U a G\+P\+U strany.

C\+P\+U strana se obvykle stará o tyto úkoly\+:
\begin{DoxyItemize}
\item Příprava dat pro kreslení (modely, textury, matice, ...)
\item Upload dat na G\+P\+U a nastavení G\+P\+U
\item Spuštění vykreslení
\end{DoxyItemize}

G\+P\+U strana je složena ze dvou částí\+: video paměti a zobrazovacího řetězce. Vykreslovací řetězec se skládá z několika částí\+:
\begin{DoxyItemize}
\item Vertex Puller
\item Vertex Processor
\item Primitive Assembly
\item Clipping
\item Rasterize
\item Fragment Processor
\item Per-\/\+Fragment Operations
\end{DoxyItemize}\hypertarget{index_terminologie}{}\subsection{Terminologie}\label{index_terminologie}
{\bfseries Vertex} je kolekce několika vertex atributů. Tyto atributy mají svůj typ a počet komponent. Každý vertex atribut má nějaký význam (pozice, hmotnost, texturovací koordináty), které mu přiřadí programátor/modelátor. Z několika vrcholů je složeno primitivum (trojúhelník, úsečka, ...)

{\bfseries Vertex atribut} je jedna vlastnost vrcholu (pozice, normála, texturovací koordináty, hmotnost, ...). Atribut je složen z 1,2,3 nebo 4 komponent daného typu (F\+L\+O\+A\+T, I\+N\+T, ...). Sémantika atributu není pevně stanovena (atributy mají pouze pořadové číslo -\/ attrib\+Index) a je na každém programátorovi/modelátorovi, jakou sémantiku atributu přidělí.  {\bfseries Fragment} je kolekce několika atributů (podobně jako Vertex). Tyto atributy mají svůj typ a počet komponent. Fragmenty jsou produkovány resterizací, kde jsou atributy fragmetů vypočítány z vertex atributů pomocí interpolace. Fragment si lze představit jako útržek původního primitiva.

{\bfseries Fragment atribut} je jedna vlastnost fragmentu (podobně jako vertex atribut).

{\bfseries Interpolace} Při přechodu mezi vertex atributem a fragment atributem dochází k interpolaci atributů. Atributy jsou váhovány podle pozice fragmentu v trojúhelníku (barycentrické koordináty).  {\bfseries Vertex Processor} (často označován za Vertex Shader) je funkční blok, který je vykonáván nad každým vertexem. Jeho vstup i výstup je Vertex. Výstupní vertex má obvykle jiné vertex atributy než vstupní vertex. Výstupní vertex má vždy atribut -\/ gl\+\_\+\+Position (pozice vertexu v clip-\/space). Vertex Processor se obvykle stará o transformace vrcholů modelu (posuny, rotace, projekce). Jelikož Vertex Processor pracuje po vrcholech, je vhodný pro efekty jako vlnění na vodní hladině, displacement mapping apod. Vertex Processor má informace pouze o jednom vrcholu v daném čase (neví nic o sousednostech vrcholů). Vertex processor je programovatelný.

{\bfseries Fragment Processor} (často označován za Fragment Shader/\+Pixel Shader) je funkční blok, který je vykonáván nad každým fragmentem. Jeho vstup i výstup je Fragment. Výstupní fragment má obykle jiné attributy. Fragment processor je programovatelný.

{\bfseries Shader} je program/funkce, který běží na některé z programovatelných částí zobrazovacího řetezce. Shader má vstupy a výstupy, které se mění s každou jeho invokací. Shader má také vstupy, které zůstávají konstantní a nejsou závislé na číslu invokace shaderu. Shaderů je několik typů, v tomto projektu se používají pouze 2 -\/ vertex shader a fragment shader. V tomto projektu jsou shadery reprezentovány pomocí standardních Cčkovských funkcí.

{\bfseries Vertex Shader} je program, který běží na vertex processoru. Jeho vstupní interface obsahuje\+: vertex, uniformní proměnné a další proměnné (číslo vrcholu gl\+\_\+\+Vertex\+I\+D, ...). Jeho výstupní inteface je vertex, který vždy obsahuje proměnnou gl\+\_\+\+Position -\/ pozici vertexu v clip-\/space.

{\bfseries Fragment Shader} je program, který běží na fragment processoru. Jeho vstupní interface obsahuje\+: fragment, uniformní proměnné a proměnné (souřadnici fragmentu ve screen-\/space gl\+\_\+\+Frag\+Coord, ...). Jeho výstupní interface je fragment.

{\bfseries Shader Program} je kolekce programů, které běží na programovatelných částech zobrazovacího řetězce. Obsahuje vždy maximálně jeden shader daného typu. V tompto projektu je program reprezentován pomocí dvou ukazatelů na funkce.  {\bfseries Buffer} je lineární pole dat ve video paměti na G\+P\+U. Do bufferů se ukládají vertex attributy vextexů modelů nebo indexy na vrcholy pro indexované vykreslování. Buffer je na C\+P\+U straně reprezentování celočíselným identifikátorem.

{\bfseries Binding point} je místo, kam lze připojit buffer. Některé binding pointy slouží pro indexování, jiné pro vertex atributy.

{\bfseries Uniformní proměnná} je proměná uložená v konstantní paměti G\+P\+U. Všechny programovatelné bloky zobrazovacího řetězce z nich mohou pouze číst. Jejich hodnota zůstává stejná v průběhu kreslení (nemění se v závislosti na číslu vertexu nebo fragmentu). Jejich hodnodu lze změnit z C\+P\+U strany pomocí funkcí jako je gl\+Uniform1f, gl\+Uniform1i, gl\+Uniform2f, gl\+Uniform\+Matrix4fv apod. Uniformní proměnné jsou vhodné například pro uložení transformačních matic nebo uložení času.

{\bfseries Vertex Puller} se stará o přípravů vrcholů. K tomuto účelu má tabulku s nastavením (vertex arrays object). Vertex puller si můžete představit jako sadu čtecích hlav. Každá čtecí hlava se stará o přípravu jednoho vertex atributu. Mezi nastavení čtecí hlavy patří\+: ukazatel na začátek bufferu, offset a krok. Vertex puller může obsahovat indexování.

{\bfseries Vertex Arrays Object (V\+A\+O)} je tabulka s nastavením vertex pulleru. Obsahuje seznam nastavení čtecích hlav pro jednotlivé vertex atributy.

{\bfseries Zobrazovací řetězec} je obvykle kus hardware na grafické kartě, který se stará o vyreslování. Grafická karta je složena ze dvou částí\+: paměti a zobrazovacího řetězce. V paměti se nacházejí buffery, textury, uniformní proměnné, programy, nastavení vertex pulleru a framebuffery. Pokud se spustí kreslení N vrcholů, je vertex puller spuštěn N krát a sestaví N vrcholů. Nad každým vrcholem je puštěn vertex shader. Výstupem vertex shaderu je nový vrchol. Blok sestavení primitiv \char`\"{}si počká\char`\"{} na 3 vrcholy z vertex shaderu (pro trojúhelník) a vloží je do jedné struktury. Blok clipping ořeže trojúhelníky pohledovým jehlanem. Následuje perspektivní dělení, které vydělí pozice vertexů homogenní složkou. Poté následuje viewport transformace, která podělené vrcholy transformuje do rozlišení obrazovky. Rasterizace trojúhelníky nařeže na fragmenty a interpoluje vertex atributy. Nad každým fragmentem je spuštěn fragment shader. Než jsou fragmenty zapsány zpět do paměti G\+P\+U (framebufferu) jsou provedeny per-\/fragment operace (zjištění viditelnosti fragmentů podle hloubky uchované v depth bufferu).  {\bfseries Uniformní lokace} je číslo, které reprezentuje jednu uniformní proměnnou.

{\bfseries Vertex atribut lokace} je číslo, které reprezentuje jeden vertex atribut.

{\bfseries gl\+\_\+\+Vertex\+I\+D} je číslo vrcholu, které je vypočítáno pomocí indexování a pořadového čísla vyvolání vertex shaderu.

{\bfseries Indexované kreslení} je způsob snížení redundance dat s využitím indexů na vrcholy.  \hypertarget{index_glsl}{}\section{Jazyk G\+L\+S\+L}\label{index_glsl}
Jazyk G\+L\+S\+L slouží pro implementaci algoritmů (shader programů), které běží na grafické kartě. Jazyk je odvozen z jazyka C a je rozšířen o vektorové a maticové typy\+: \begin{DoxyVerb}* vec3 a;//vektor třísložkový, typ float
* ivec4 b;//vektor čtyřsložkový, typ int
* mat4 c;//matice 4x4, typ float
* \end{DoxyVerb}
 Jazyk má přetížené operátory a funkce pro práci s těmito novými typy\+: \begin{DoxyVerb}* vec3 a = vec3(1,2,3);
* vec3 b = vec3(2,3,2);
* b += a;//pricte k "b" "a" po komponentách
* b.x = 10;//nahraje do x složky 10
* b[1] = 11;//nahraje do y složky 11
* b.xy = vec2(13,14);//nahraje do x 13 a do y 14
* b = a.xxy;// je akvivalentni k b.x = a.x; b.y = a.x; b.z = a.y;
* float s = dot(a,b);//skalární součin
* b *= 10;//vynásobí všechny složky vektoru 10
* mat3 m;
* b = m*a;//násobení maticí
* \end{DoxyVerb}
 Jazyk vynucuje, aby první řádek zdrojového kódu obsahoval verzi \begin{DoxyVerb}* #version 450
* \end{DoxyVerb}
 Zdrojový kód shaderu je složen z interface a funkce main. Interface obsahuje deklarace globálních proměnných, pomocí kterých shader komunikuje s okolím. Funkce main je vykonána nad každým vertexem/fragmentem. Kvalifikátor proměnné \char`\"{}uniform\char`\"{} říká, že data proměnné budou uložena v konstantní paměti G\+P\+U, která lze změnit z C\+P\+U. \begin{DoxyVerb}* uniform mat4 viewMatrix;
* \end{DoxyVerb}
 Kvalifikátor \char`\"{}in\char`\"{} a \char`\"{}out\char`\"{} říkají, že data proměnné přijdou z předcházejícího programovatelného bloku nebo vertex pulleru nebo odejdou do následujícího programovatelného bloku nebo do per-\/fragment operací. Lokaci lze získat pouze z uniformních proměnných a \char`\"{}in\char`\"{} proměnných vertex shaderu. Takto dekorované proměnné mění svoji hodnotu per-\/vertex nebo per-\/fragment. Jazyk obsahuje spoustu vestavěnách funkcí\+: \begin{DoxyVerb}* vec3 a,b;
* dot(a,b);//skalární součin
* cross(a,b);//cross product
* reflect(a,b);//a je odražen podle b
* normalize(a);//normalizace
* clamp(f,a,b) ořez proměnné f do rozsahu [a,b]
* min(a,b) návrat menší hodnoty
* max(a,b) návrat větší hodnoty
* \end{DoxyVerb}
\hypertarget{index_sestaveni}{}\section{Sestavení}\label{index_sestaveni}
Linux v C\+V\+T\+: pustit skript linux\+\_\+create\+\_\+makefile.\+sh -\/ to vám vytvoří makefile ve složce build Postup\+:
\begin{DoxyEnumerate}
\item \$ ./linux\+\_\+create\+\_\+makefile.sh
\item \$ cd build
\item \$ make
\end{DoxyEnumerate}

Windows\+: V C\+V\+T je naistalována stará verze C\+Make, stáhněte {\bfseries novou} verzi C\+Make z\+: \href{http://www.fit.vutbr.cz/~imilet/shared/cmake-3.11.1-win64-x64.zip}{\tt cmake na kazi}, nebo z \href{https://cmake.org/files/v3.11/cmake-3.11.1-win64-x64.zip}{\tt C\+M\+A\+K\+E}. Rozbalte cmake a spusťte program {\bfseries bin/cmake-\/gui.\+exe}.  Cvičení bylo testováno na Ubuntu 14.\+04, Ubuntu 16.\+04, Visual Studio 2013, Visual Studio 2015, Visual Studio 2017. Cvičení vyžaduje 64 bitové sestavení. Cvičení využívá build systém \href{https://cmake.org/}{\tt C\+M\+A\+K\+E}. C\+Make je program, který na základně konfiguračních souborů \char`\"{}\+C\+Make\+Lists.\+txt\char`\"{} vytvoří \char`\"{}makefile\char`\"{} v daném vývojovém prostředí. Dokáže generovat makefile pro Linux, mingw, solution file pro Microsoft Visual Studio, a další. Pro Visual Studio 2017 je potřeba nový C\+M\+A\+K\+E 3.\+10.\+1 a vyšší. Postup\+:
\begin{DoxyEnumerate}
\item stáhnout projekt
\item rozbalit projekt
\item ve složce build spusťte \char`\"{}cmake-\/gui ..\char`\"{} případně \char`\"{}ccmake ..\char`\"{}
\item vyberte si překladovou platformu (64 bit).
\item configure
\item generate
\item make nebo otevřete vygenerovnou Microsoft Visual Studio Solution soubor.
\end{DoxyEnumerate}

Cvičení vyžaduje pro sestavení knihovnu \href{https://www.libsdl.org/download-2.0.php}{\tt S\+D\+L2}. Ta je pro 64bit build ve Visual Studiu přibalena. Cesty na hlavičkové soubory a libs jsou nastaveny pomocí checkboxu U\+S\+E\+\_\+\+P\+R\+E\+B\+U\+I\+L\+D\+\_\+\+L\+I\+B\+\_\+\+P\+A\+C\+K\+A\+G\+E. Pod Linuxem si stáhněte S\+D\+L2 zdrojáky, zkompilujte je (s pomocí C\+M\+A\+K\+E) a nainstalujte knihovnu.\hypertarget{index_spousteni}{}\section{Spouštění}\label{index_spousteni}
Cvičení je možné po úspěšném přeložení pustit přes aplikaci {\bfseries izg2017\+Lab6}.\hypertarget{index_ovladani}{}\section{Ovládání}\label{index_ovladani}
Program se ovládá pomocí myši a klávesnice\+:
\begin{DoxyItemize}
\item stisknuté levé tlačítko myši + pohyb myší -\/ rotace kamery
\item stisknuté pravé tlačítko myši + pohyb myší -\/ přiblížení kamery
\item \char`\"{}n\char`\"{} -\/ přepne na další scénu/metodu \char`\"{}p\char`\"{} -\/ přepne na předcházející scénu/metodu 
\end{DoxyItemize}