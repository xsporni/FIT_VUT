\hypertarget{group__task3}{}\section{Třetí úkol}
\label{group__task3}\index{Třetí úkol@{Třetí úkol}}
\begin{DoxyRefDesc}{Todo}
\item[\hyperlink{todo__todo000002}{Todo}]3.\+1.) Upravte vertex shader. Vašim úkolem je přidat druhý vertex atribut -\/ normálu vrcholu. Dále přidejte dvě výstupní proměnné typu vec3 a zapište do nich pozici a normálu vrcholu ve world-\/space. Tyto proměnné budete potřebovat pro výpočet osvětlení ve fragment shaderu. Upravujte phong\+Vertex\+Shader\+Source proměnnou. \end{DoxyRefDesc}


\begin{DoxyRefDesc}{Todo}
\item[\hyperlink{todo__todo000003}{Todo}]3.\+2.) Upravte fragment shader (proměnná phong\+Fragment\+Shader\+Source). Vašim úkolem je implementovat phongův osvětlovací model. Přidejte dvě vstupní proměnné (typ vec3) stejného názvu, jako nově přidané výstupní proměnné ve vertex shaderu. V jedné obdržíte pozice fragmentu ve world-\/space. V druhé obdržíte normálu fragmentu ve world-\/space. Dále přidejte dvě uniformní proměnné (typ vec3) pro pozici kamery a pro pozici světla. Difuzní barvu materiálu nastavte podle zadání konkrétní skupiny cvičení, ptejte se pokud nevíte jak má materiál vypadat Spekulání barvu materiálu nastavte na vec3(1.\+f,1.\+f,1.\+f) -\/ bílá.~\newline
 Shininess faktor nastavte na 40.\+f.~\newline
 Předpokládejte, že světlo má bílou barvu.~\newline
 Barva se vypočítá podle vzorce d\+F$\ast$d\+M$\ast$d\+L + s\+F$\ast$s\+M$\ast$s\+L.~\newline
 d\+M,s\+M jsou difuzní/spekulární barvy materiálu -\/ vektory.~\newline
 d\+L,s\+L jsou difuzní/spekulární barvy světla -\/ vektory.~\newline
 d\+F,s\+F jsou difuzní/spekulární faktory -\/ skaláry.~\newline
 d\+F lze vypočíst pomocí vztahu clamp(dot(\+N,\+L),0.\+f,1.\+f) -\/ skalární součin a ořez do rozsahu \mbox{[}0,1\mbox{]}.~\newline
 N je normála fragmentu (nezapomeňte ji normalizovat).~\newline
 L je normalizovaný vektor z pozice fragmentu směrem ke světlu.~\newline
 s\+F lze vypočíst pomocí vztahu pow((clamp(dot(\+R,\+L),0.\+f,1.\+f)),s) -\/ skalární součin, ořez do rozsahu \mbox{[}0,1\mbox{]} a umocnění.~\newline
 s je shininess faktor.~\newline
 R je odražený pohledový vektor V; R = reflect(\+V,\+N).~\newline
 V je normalizovaný pohledový vektor z pozice kamery do pozice fragmentu.~\newline
 ~\newline
 Nezapomeňte, že programujete v jazyce G\+L\+S\+L, který zvládá vektorové operace.~\newline
 {\bfseries Seznam užitečných funkcí\+:}
\begin{DoxyItemize}
\item dot(x,y) -\/ funkce vrací skalární součin dvou vektorů x,y stejné délky
\item clamp(x,a,b) -\/ funkce vrací ořezanou hodnotu x do intervalu \mbox{[}a,b\mbox{]}
\item normalize(x) -\/ funkce vrací normalizovaný vektor x
\item reflect(\+I,\+N) -\/ funkce vrací odražený vektor I podle normály N
\item pow(x,y) -\/ funkce vrací umocnění x na y -\/ x$^\wedge$y 
\end{DoxyItemize}\end{DoxyRefDesc}
\begin{DoxyRefDesc}{Todo}
\item[\hyperlink{todo__todo000006}{Todo}]3.\+3.) Ve funkci \hyperlink{student_8h_ac2adb2ba4e748239b9db4d037584d3cc}{phong\+\_\+on\+Init()} získejte lokace přidaných uniformních proměnných pro pozici světla a pro pozice kamery. Zapište lokace do příslušných položek ve struktuře \hyperlink{structPhongVariables}{Phong\+Variables}. Nezapomeňte, že lokace získáte pomocí jména proměnné v jazyce G\+L\+S\+L, které jste udělali v předcházejícím kroku.~\newline
 {\bfseries Seznam funkcí, které jistě využijete\+:}
\begin{DoxyItemize}
\item gl\+Get\+Uniform\+Location 
\end{DoxyItemize}\end{DoxyRefDesc}


\begin{DoxyRefDesc}{Todo}
\item[\hyperlink{todo__todo000007}{Todo}]3.\+4.) Ve funkci \hyperlink{student_8h_ac2adb2ba4e748239b9db4d037584d3cc}{phong\+\_\+on\+Init()} nastavte druhý vertex atribut pro normálu podobně jako pro pozici. Musíte získat lokaci vstupní proměnné ve vertex shaderu, kterou jste přidali v předcházejícím kroku. Musíte správně nastavit stride a offset -\/ normála nemá nulový offset.~\newline
 \end{DoxyRefDesc}


\begin{DoxyRefDesc}{Todo}
\item[\hyperlink{todo__todo000010}{Todo}]3.\+5.) Ve funkci \hyperlink{student_8h_a53ffbb1a271d285abdaf7a029192f47e}{phong\+\_\+on\+Draw()} nahrajte pozici světla a pozici kamery na G\+P\+U. Pozice světla a pozice kamery je v proměnných phong.\+light\+Position a camera\+Position.~\newline
 {\bfseries Seznam funkcí, které jistě využijete\+:}
\begin{DoxyItemize}
\item gl\+Uniform3f nebo gl\+Uniform3fv 
\end{DoxyItemize}\end{DoxyRefDesc}
