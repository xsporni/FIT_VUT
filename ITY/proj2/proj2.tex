\documentclass[a4paper, 11 pt, twocolumn]{article}
\usepackage[left=1.5cm, text={18cm, 25cm}, top=2.5cm]{geometry}
\usepackage[utf8]{inputenc}
\usepackage[czech]{babel}
\usepackage[IL2]{fontenc}
\usepackage{times}
\usepackage{amsthm}
\usepackage{amsmath}
\usepackage{amsfonts}



\theoremstyle{definition}
\newtheorem{theorem}{Definice}

\newtheorem{pov}{Věta}

\begin{document}
\begin{titlepage}
    \begin{center}
      \Huge
      \textsc{Fakulta informačních technologií \\[0.3em]
      Vysoké učení technické v~Brně}
      \vspace{\stretch{0.382}}
      \LARGE\\{Typografie a publikování \,--\, 2. projekt \\[0.4em]
             Sazba dokumentů a~matematickýh výrazů}
      \vspace{\stretch{0.618}}
    \end{center}
    \Large{2018 \hfill Alex Sporni}
  \end{titlepage}
  
\section*{Úvod}
V~této úloze si vyzkoušíme sazbu titulní strany, matematic\-kých
vzorců, prostředí a~dalších textových struktur obvyklých
pro technicky zaměřené texty (například rovnice (\ref{eq:rovnica1})
nebo Definice \ref{veta_1} na straně \pageref{veta_1}). Rovněž si vyzkoušíme používání
odkazů \verb|\ref| a~\verb|\pageref|.

Na titulní straně je využito sázení nadpisu podle optického
středu s~využitím zlatého řezu. Tento postup byl
probírán na přednášce. Dále je použito odřádkování se
zadanou relativní velikostí 0.4em a 0.3em.

\section{Matematický text}
Nejprve se podíváme na sázení matematických symbolů
a~výrazů v~plynulém textu včetně sazby definic a~vět s~využitím
balíku \texttt{amsthm}. Rovněž použijeme poznámku pod
čarou s~použitím příkazu \verb|\footnote|. Někdy je vhodné
použít konstrukci \verb|${}$|, která říká, že matematický text
nemá být zalomen.
\begin{theorem}\label{veta_1}
Turingův stroj \emph{(TS) je definiván jako šestice tvaru}
$M=(Q,\Sigma,\Gamma,\delta,q_0,q_F)$, \emph{kde:}

\begin{itemize}
\item $Q$ \emph{je konečná množina} vnitřních (řídicích) stavů,
\item $\Sigma$ \emph{je konečná množina symbolů nazývaná} vstupní
abeceda, $\Delta \not\in \Sigma$,
\item $\Gamma$ \emph{je konečná množina symbolů}, $\Sigma \subset \Gamma $, $\Delta \in \Gamma$, \emph{nazývaná} pásková abeceda,
\item $\delta$ : $(Q \setminus\{q_F\})\times\Gamma \rightarrow Q\times(\Gamma\cup\{L,R\})$ \emph{kde} $L,R \not\in\Gamma$, \emph{je parciální} přechodová funkce,
\item $q_0$ \emph{je} počáteční stav, $q_0 \in Q$ \emph{a}
\item $q_F$ \emph{je} koncový stav, $q_F \in Q$.
\end{itemize}

Symbol $\Delta$ značí tzv. \emph{blank} (prázdný symbol), který
se vyskytuje na místech pásky, která nebyla ještě použita
(může ale být na pásku zapsán i~později).

\emph{Konfigurace pásky} se skládá z~nekonečného řetězce,
který reprezentuje obsah pásky, a~pozice hlavy na tomto
řetězci. Jedná se o~prvek množiny $\{\gamma\Delta^\omega \mid \gamma\in\Gamma^*\}\times \mathbb{N}$.\footnote{Pro libovolnou abecedu $\Sigma$ je $\Sigma^\omega$ množina všech \emph{nekonečných} řetězců nad $\Sigma$, tj. nekonečných posloupností symbolů ze $\Sigma$. Pro připomenutí: $\Sigma^*$ je množina všech \emph{konečných} řetězců nad $\Sigma$.} \emph{Konfiguraci pásky} obvykle zapisujeme jako $\Delta xyz\underline{z}x\Delta...$
(podtržení značí pozici hlavy). \emph{Konfigurace stroje} je pak
dána stavem řízení a konfigurací pásky. Formálně se jedná
o~prvek množiny $Q\times\{\gamma\Delta^\omega\mid\gamma\in\Gamma^*\}\times\mathbb{N}$.
\end{theorem}
\subsection{Podsekce obsahující větu a odkaz}
\begin{theorem}
\label{veta_2}
Řetězec $w$ nad abecedou $\Sigma$ je přijat TS \emph{$M$
jestliže $M$ při aktivaci z~počáteční konfigurace pásky 
$\underline{\Delta}w\Delta...$ a~počátečního stavu $q_0$ zastaví
přechodem do koncového stavu $q_F$, tj. $(q_0,\Delta,w\Delta^\omega,0)\underset{M}{\overset{*}{\vdash}}(q_F,\gamma,n)$ pro nějaké $\gamma\in\Gamma^*$ a~$n\in\mathbb{N}$.}

\emph{$L(M)=\{w\;|\;w$ je přijat TS $M\} \subseteq \Sigma^*$ nazý-
váme jazyk přijímaný TS $M$.} 
\end{theorem}

Nyní si vyzkoušíme sazbu vět a~důkazů opět s~použitím
balíku \verb|amsthm|.


\begin{pov}
\emph{Třída jazyků, které jsou přijímány TS, odpovídá} rekurzivně vyčíslitelným jazykům.
\end{pov}

\begin{proof}
V~důkaze vyjdeme z~Definice \ref{veta_1} a~\ref{veta_2}.
\end{proof}

\section{Rovnice a odkazy}
Složitější matematické formulace sázíme mimo plynulý
text. Lze umístit několik výrazů na jeden řádek, ale pak je
třeba tyto vhodně oddělit, například příkazem \verb|\quad|.

$$\sqrt[i]{x^3_i} \quad \text{kde } x_i \text{ je } i\text{-té  sudé číslo}\quad y^{2\cdot y_i}_i \not=y_i^{y_i^{y_i}} $$

V~rovnici (\ref{eq:rovnica1}) jsou využity tři typy závorek s~různou explicitně definovanou velikostí.

\begin{eqnarray}
\label{eq:rovnica1}
	x & = &\bigg\{ \Big(\big[a + b\big] * c\Big)^d \oplus 1 \bigg\}\\
	y & = &\lim_{x\to\infty} \frac{\sin^2x + \cos^2x}{\frac{1}{\log_{10}x}} \nonumber
\end{eqnarray}

V~této větě vidíme, jak vypadá implicitní vysázení limity $lim_{n\to\infty}f(n)$v~normálním odstavci textu. Podobně je to i~s~dalšími symboly jako $\sum^{n}_{i=1} 2^{i}$
či $\bigcup_{A\in\mathcal{B}}A$. V~případě vzorců $\lim\limits_{x\to\infty} f(n)$ a~$\sum\limits_{i=1}^{n}2^{i}$ jsme si vynutili méně úspornou sazbu příkazem \verb|\limits|.
\begin{eqnarray}
\int\limits^b_a f(x) \, \mathrm{d}x & = & - \int^a_b g(x) \, \mathrm{d}x \\
\overline{\overline{A \vee B}} & \Leftrightarrow & \overline{\overline{A} \wedge \overline{B}}
\end{eqnarray}

\section{Matice}
Pro sázení matic se velmi často používá prostředí \verb|array| a~závorky (\verb|\left|, \verb|\right|).

$$ 
\left
(\begin{array}{ccc}
a+b & \widehat{\xi + \omega} & \hat{\pi} \\
\vec{a} & \overleftrightarrow{AC} & \beta \\
\end{array}
\right)
= 1 \Longleftrightarrow \mathbb{Q} = \mathbb{R}
$$

$$
\mathbf{A} =
\left\|
\begin{array}{cccc}
a_{11} & a_{12} & \ldots & a_{1n} \\
a_{21} & a_{22} & \ldots & a_{2n} \\
\vdots & \vdots & \ddots & \vdots \\
a_{m1} & a_{m2} & \ldots & a_{mn}

\end{array}
\right\| = \left|
		\begin{array}{cc}
			t & u\\
			v & w
		\end{array}
		\right| = tw - uv
$$
Prostředí \verb|array| lze úspěšně využít i~jinde.
$$
\binom{n}{k} =
\left\{
\begin{array}{ll}
	\frac{n!}{k! (n - k)!} & \text{pro } 0 \leq k\leq n \\
	0 & \text{pro } \;k < 0 \text{ nebo } k > n
\end{array}
\right.
$$ 
\section{Závěrem}
V~případě, že budete potřebovat vyjádřit matematickou
konstrukci nebo symbol a~nebude se Vám dařit jej nalézt
v~samotném {\LaTeX}u, doporučuji prostudovat možnosti balíku
maker \AmS-\LaTeX.
\end{document}
  
  

