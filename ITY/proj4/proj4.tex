\documentclass[a4paper,11pt]{article}
\usepackage[text={17cm,24cm},left=2cm,top=3cm]{geometry}
\usepackage[czech]{babel}
\usepackage[utf8]{inputenc}
\usepackage[IL2]{fontenc}
\usepackage[hyphens]{url}
\def\UrlBreaks{\do\/\do-}


\begin{document}
\begin{titlepage}
		\begin{center}
			\Huge
			\textsc{Vysoké učení technické v~Brně} \\
			\huge
			\textsc{Fakulta informačních technologií} \\
			\vspace{\stretch{0.382}}
			\LARGE
			Typografie a~publikování\,--\,4.~projekt \\
			\Huge
			Bibliografické citácie
			\vspace{\stretch{0.618}}
		\end{center}

		{\Large
			\today
			\hfill
			Alex Sporni
		}
\end{titlepage}

\section{Úvod}
\emph{\uv{Jednou z~oblasti lidské činnosti, kterou výrazně poznamenalo rozšírení osobních počítačů, je zpracování textu.}} 
\cite{Rybicka2003}
\section{Systém \LaTeX}
\LaTeX\ je komplexná sada príkazov, ktoré sú využívané prepacovaným sádzacím programom \TeX. Tento program slúži na prípravu dokumentov, vedeckých článkov, hlásení a~komplexných kníh. \TeX ako aj \LaTeX\ je open source software čo znamená, že je úplne zadarmo. \cite{Kopka2004}
\section{Prečo si zvoliť {\LaTeX} a~prečo je praktické ho ovládať}
Oproti editorom ako sú Word a~Open Office {\LaTeX} nijako neobmedzuje. Je jednoduchý, rýchly a~ponúka možnosť presnej citácie podľa normy ISO 690 \cite{Martinek2013}. Medzi ďalšie výhody patria:
\begin{itemize}
\item jednoduché vysádzanie matematických znakov
\item možnosť si nadefinovať ľubovolné odsadenie alebo vzorec
\item text je prehľadný a jasne štruktúrovaný
\item \LaTeX je oproti iným konkurenčým produktom úplne zdarma \cite{Found2013}
\end{itemize}
\section{Štruktúra dokumentu}
Medzi najčastejšie chyby bežných editorov patria \textit{vdova} a~\textit{sirota}. 
Vdovou nazývame prvý riadok nového odstavca, ktorý zostal na predošlej strane a sirotu nazývame posledný riadok odstavca, ktorý pretiekol na ďalšiu stranu, kde sa stal prvým riadkom. 
Obe spomenuté chyby sa v~kvalitných prácach nesmú vyskytnúť. Okrem toho, že sú na oko nepríjemné, tak hrubo porušujú  pravidlá správnej sadzby a pri čítaní tak človek stratí kontinuitu a hlavnú tému. Jednoduchý príkaz na členenie paragrafov je \verb|\par|. \cite{root2001}
\section{Matematické vzorce v~{\LaTeX}e}
Sadzba matematiky je samostatnou doménou {\LaTeX}u. Pre sadzbu matematiky existujú rôzne prostredia. Jedným z~týchto prostredí je prostredie \texttt{math}, ktoré umožňuje vysádzať vzorce vo~vnútri textu. Vzorce sa uzavierajú do príkazov \verb|\begin{math}| a~\verb|\end{math}|.
\newline Jednoduchá ukážka: \begin{math}s=\frac{1}{2}at^2\end{math}, zdrojový kód: \verb|\begin{math}s=\frac{1}{2}at^2\end{math}|
{\LaTeX} si jednoducho poradí aj s~oveľa zložitejšími vzorcami, ktoré by vo~worde bolo veľmi obtiažne vysádzať, v~niektorých prápadoch priam nemožné. \cite{Bojko2011}
$$
f=
\sin^2\alpha\frac{
\int\limits_1^2
\sqrt{1+\left(\frac{(x^6-1)^2}{2x^3}\right)^3}
dx}
{\frac{3x}{x^2}}+
(\sin\beta - \cos\alpha)
$$
\section{Tvorba Tabuliek}
Tabuľky sú jedným zo spôsobov ako prehľadne zobraziť dáta. \LaTeX nám nám tú možnosť ponúka prostredím \texttt{tabular}. Prostredie \texttt{tabular} má povinný a nepovinný parameter. Nepovinný parameter slúži na určenie vertikálneho zarovnania. Povinný parameter určuje horizontálne zarovnanie v~jednotlivých stĺpcoch. \cite{Cerny2011}
\section{Slides}
Trieda \texttt{slides} ponúka základnú sadu príkazov pre tvorbu a~prácu so snímkami prezentácie.
Samotná trieda sa volá príkazom \verb|\documentclass{slides}|. 
Jednotivé snímky prezentácie sa umiestnia medzi párovu sadu príkazov \verb|\begin{document}| a~\verb|\end{document}|. \cite{Olsak2014}
\section{HTML}
HTML je masívne využívané ako rozhranie ktoré poskytuje služby používateľom. Weboví vývojári vytvárajú a~editujú stránky s~vysokým tempom a~snažia sa podporovať najnovšie normy a~štandardy HTML.\cite{Mendes2018}
Existujú isté snahy konzorcia World Wide Web vyvýjať HTML 5 ako jediný štandard, ktorý má za úlohu poskytovať vylepšené funkcie, bez proprietárnych technológii. \cite{Nichols2010}


\newpage
\bibliographystyle{czechiso}
\renewcommand{\refname}{Literatúra}
\bibliography{proj4}

\end{document}